\chapter{Evaluasi dan Analisa ABS MVC Framework}

Bab ini berisi tentang hasil analisis dan evaluasi dari MVC Framework yang telah dibuat. Dalam melakukan proses evaluasi, penulis membandingkan proses yang dilakukan dalam membuat perangkat lunak berbasis web antara menggunakan ABS MVC Framework dengan menggunakan framework lain. Adapun aplikasi web yang digunakan dalam proses evaluasi ini adalah aplikasi web yang telah penulis buat pada bagian penerapan ABS MVC Framwork (aplikasi katalog produk). Framework lain yang penulis gunakan dalam proses evaluasi ini adalah Play Framework (JAVA MVC Framework) versi 2.3 \footnote{https://playframework.com/download}. Berikut ini adalah poin-poin yang akan dievaluasi pada bab ini:

\begin{enumerate}
    \item Evaluasi proses penerapan Model-View-Controller.
    \item Evaluasi proses penanganan perubahan \textit{requirement}.
    \item Evaluasi penerapan SPL dengan menggunakan ABS MVC Framework.
\end{enumerate}

\section{Evaluasi Penerapan Model-View-Controller}

Dalam proses pengembangan apalikasi web dengan menggunakan MVC \textit{framework}, setiap kode yang dibuat akan dikelompokan kedalam komponen Model-View-Controller. Oleh karena itu, untuk mengevalusi framework ABS MVC yang dibuat penulis membandingan proses pembuatan komponen-komponen MVC tersebut antara Play Framework dengan ABS MVC Framework. Berikut adalah hasil evaluasi dan analisa ABS MVC Framework dari poin penerapan MVC-nya.

\subsection{Proses Pembuatan Komponen Model}

Komponen model merupakan komponen yang merepresentasikan entitas data pada aplikasi. Dalam banyak MVC Framework, komponen model dapat diintegrasikan dengan peragkat lunak basis data sehingga setiap komponen model yang dibuat akan merepresentasikan setiap tabel yang ada di dalam basis data. Dalam konteks ABS MVC Framework, proses integrasi komponen model dengan perangkat lunak basis data belum dapat dilakukan seperti yang sudah dibahas pada bagian batasan penelitian. Berikut ini adalah perbandingan antara komponen Model yang dibuat dengan menggunakan ABS MVC Framework dengan Play Framework.

\begin{lstlisting}[
caption=Komponen Model menggunakan ABS,
label={lst:modelUsingABS}
]
interface Product
{
    ...
}

class ProductImpl 
	(String sku, String name, String description, String price) implements Product
{
	String getSku() { return this.sku; }
	String getName() { return this.name; }
	String getDescription() { return this.description; }
	String getPrice() { return this.price; }
	Unit setSku(String sku) { this.sku = sku; }
	Unit setName(String name) { this.name = name; }
	Unit setDescription(String description) { this.description = description; }
	Unit setPrice(String price) { this.price = price; }
}
\end{lstlisting}

\begin{lstlisting}[
caption=Komponen Model menggunakan Play Framework,
label={lst:modelUsingPlay},
escapeinside={!}{!}
]
@Entity !\label{lst:modelAnotasi}!
public class Product extends Model
{
	@Id !\label{lst:modelAnotasi2}!
	public Long id;
	public String sku;
	public String name;
	public String description;
	public Long price;

	public Product(String sku, String name,
		String description, Long price)
	{
		this.sku = sku;
		this.name = name;
		this.description = description;
		this.price = price;
	}

	public static Finder<Long,Product> find = new Finder<Long,Product>( !\label{lst:modelPlayFinder}!
    Long.class, Product.class); 
}
\end{lstlisting}

Seperti yang terlihat pada kode \ref{lst:modelUsingABS} dan \ref{lst:modelUsingPlay} diatas, kedua buah komponen Model tersebut tidak memiliki perbedaan yang signifikan. keduanya hanya berisikan atribut-atribut yang merepresentasikan data produk pada aplikasi web yang dibuat. Hanya saja, pada komponen model yang dibuat dengan menggunakan Play Framework memiliki tambahan anotasi (bari \ref{lst:modelAnotasi} dan \ref{lst:modelAnotasi2}) serta tambahan \textit{method} (baris \ref{lst:modelPlayFinder}) yang digunakan oleh Play Framework untuk mengintegrasikan komponen tersebut dengan perangkat lunak basis data.\\

Walaupun terdapat beberapa perbedaan antara komponen model yang dibuat dengan ABS Framework dengan Play Framework, akan tetapi kedua komponen tersebut masih memiliki karakteristik yang sama yaitu hanya terdiri dari atribut data dan tidak mengandung proses bisnis aplikasi. Hal ini sejalan dengan definisi komponen Model yang dipaparkan oleh \cite{krasner1988desc} dan \cite{leff2001web} pada publikasi penelitiannya. Dengan demikian, penulis berkesimpulan bahwa metode yang digunakan pada ABS MVC Framework sudah sesuai dengan karakteristik komponen model pada \textit{framework} yang umum digunakan dan juga sudah sesuai dengan landasan teori yang ada.

\subsection{Proses Pembuatan Komponen View}

Dalam banyak \textit{framework} MVC, pembuatan komponen View umumnya menggunakan sintaks HTML dengan tambahan sintaks \textit{template engine} untuk dapat mengintegrasikan antara komponen View tersebut dengan komponen Model yang dibuat. Dalam konteks ABS MVC Framework, komponen View yang dibuat sudah menggunakan sintaks HTML dan \textit{template engine} Thymeleaf sedangkan Play Framewerk juga sudah menggunakan HTML dan \textit{template engine} Twirl. Berikut ini adalah perbandingan komponen View yang dibuat dengan menggunakan ABS MVC Framework dan Play Framework.

\begin{lstlisting}[
caption=Halaman \texttt{list.html} yang dibuat menggunakan Thymeleaf,
label={lst:viewThymeleaf},
escapeinside={!}{!}
]
...
<tbody>
	<tr th:each="product: ${dataList}"> !\label{lst:thymeDataList}!
		<td th:text="${#ids.seq('')}"></td>
		<td th:text="${product.sku}"></td>
		<td th:text="${product.name}"></td>
		<td th:text="${product.price}"></td>
		<td>
			<a th:href="@{http://localhost:8080/product/update.abs(sku=${product.sku})}">update</a>&nbsp;&nbsp;
			<a th:href="@{http://localhost:8080/product/delete.abs(sku=${product.sku})}">delete</a>
		</td>
	</tr>
</tbody>
...
\end{lstlisting}

\begin{lstlisting}[
caption=Halaman \texttt{list.html} yang dibuat menggunakan Twirl,
label={lst:viewTwirl},
escapeinside={!}{!}
]
@(products: List[Product]) !\label{lst:twirlDefineVariable}!
...
<tbody>
	@for((product, index) <- products.zipWithIndex){
		<tr>
			<td>@{index+1}</td>
			<td>@product.sku</td>
			<td>@product.name</td>
			<td>@product.price</td>
			<td>
				<a href="/product/update/@product.sku">update</a>&nbsp;&nbsp;
				<a href="/product/delete/@product.sku">delete</a>
			</td>
		</tr>
	}
</tbody>
...
\end{lstlisting}

Seperti yang terlihat pada kode \ref{lst:viewThymeleaf} dan \ref{lst:viewTwirl}, kedua komponen View tersebut tidak jauh berbeda dikarenakan keduanya dibuat dengan menggunakan sintaks HTML. Hal yang membedakan dari kedua komponen tersebut terletak pada sintaks \textit{template engine} yang digunakan seperti misalnya pada ABS MVC Framework menggunakan \texttt{\$\{data.sku\}} untuk mendapatkan nomor sku dari objek model yang diberikan, sedangkan Play Framework menggunakan sintaks \texttt{@product.sku}. Perbedaan sintaks ini adalah wajar dikarenakan \textit{template engine} yang digunakan oleh kedua \textit{framework} tersebut adalah berbeda.\\

Walaupun kedua komponen di atas hampir sama, namum komponen View yang dibuat oleh komponen ABS MVC Framework banyak memiliki keterbatasan jika dibandingkan dengan komponen View yang dibuat dengan menggunakan Play Framework. berikut adalah beberapa keterbatasan yang dimiliki oleh ABS MVC Framework:

\begin{itemize}
    \item Memerlukan konvensi khusus untuk dapat memanggil komponen model yang diberikan seperti misalnya penggunaan variable \texttt{dataList} (kode \ref{lst:viewThymeleaf} barsi \ref{lst:thymeDataList}) untuk objek model berbentuk list dan \texttt{data} (kode \ref{lst:htmlUpdateProduct}) untuk objek model yang berdiri sendiri. Hal ini berbeda dengan komponen View pada Play Framework yang dapat didefinisikan sendiri oleh pengembang (kode \ref{lst:viewTwirl} baris \ref{lst:twirlDefineVariable}).
    \item Komponen View pada ABS MVC Framework belum dapat menangani banyak komponen Model yang artinya ketika pengembang membuatuhkan Model Product dan Customer untuk ditampilkan di halaman web,  hal ini belum dapat dilakukan. Berbeda dengan komponen View pada Play Framework yang dapat menerima banyak model dengan menggunakan \texttt{@\(product: Product, customer: Customer\)}.
\end{itemize}

Berdasarkan hasil evaluasi diatas, penulis menyimpulkan bahwa untuk komponen View yang dibuat menggunakan ABS MVC Framework, sudah memiliki karakteristik yang sama dengan komponen yang dibuat menggunakan Play Framework. Akan tetapi, masih terdapat beberapa keterbatasan yang dimiliki sehingga tingkat fleksibilitasnya masih rendah dibandingkan dengan komponen View yang dibuat dengan menggunakan Play Framework.

\subsection{Proses Pembuatan Komponen Controller}

Secara umum fungsi dari komponen Controller adalah mengintegrasikan komponen Model dan View agar dapat dihasilkan sebuah halaman web yang dapat ditampilkan ke pengguna aplikasi. Dalam implementasinya, komponen Controller berisi bisnis proses aplikasi yang didalamnya melakukan pemrosesan \textit{input} yang diberikan oleh pengguna aplikasi. Dalam konteks ABS MVC Framework, kedua hal tersebut sudah dapat dilakukan walaupun dengan mekanisme yang sedikit berbeda dengan komponen Controller yang dibuat dengan menggunakan Play Framework. Berikut ini adalah cuplikan \textit{method} \texttt{saveUpdateProduct} yang ada di dalam Controller ABS MVC Framework dan Play Framework:

\begin{lstlisting}[
caption=Method \texttt{saveUpdateProduct} menggunakan ABS MVC Framework,
label={lst:saveUpdateABS}
]
...
Pair<String, List<Product>> saveUpdateProduct(ABSHttpRequest request)
{
	ProductDB db = new local ProductDBImpl();
	db.init();
	
	String sku = request.getInput("product_sku");
	String name = request.getInput("product_name");
	String description = request.getInput("description");
	String price = request.getInput("price");
	
	Product p = db.findBySku(sku);
	p.setName(name);
	p.setDescription(description);
	p.setPrice(price);
	db.update(p);
	
	List<Product> products = db.findAll();
	return Pair("product/list", products);
}
...
\end{lstlisting}

\begin{lstlisting}[
caption=Method \texttt{saveUpdateProduct} menggunakan Play Framework,
label={lst:saveUpdatePlay}
]
public static Result saveUpdateProduct()
{
    DynamicForm requestData = Form.form().bindFromRequest();
    String sku = requestData.get("product_sku");
    String name = requestData.get("product_name");
    String description = requestData.get("description");
    Long price = Long.parseLong(requestData.get("price"));

    List<Product> products = Product.find.where().eq("sku", sku).findList();
    Product product = products.get(0);
    product.name = name;
    product.description = description;
    product.price = price;
    product.update();

    return ok(list.render(Product.find.all()));
}
\end{lstlisting}

Seperti yang terlihat pada kode \ref{lst:saveUpdateABS} dan \ref{lst:saveUpdatePlay} diatas, secara umum apa yang dilakukan oleh kedua buah \textit{method} tersebut adalah sama yaitu (1) menerima input dari pengguna aplikasi, (2) melakukan pencarian kedalam basis data untuk mendapatkan produk dengan nomor sku yang sesuai, (3) Mengubah data produk tersebut dengan data yang diberikan oleh pengguna aplikasi dan (4) menampilkan seluruh data produk kedalam halaman \texttt{list.html}. Satu hal yang membedakan kedua komponen Controller diatas adalah pada ABS MVC Framework basis data yang digunakan masih merupakan basis data \textit{dummy} sedangkan pada Play Framework, penulis sudah menggunakan MySQL sebagai basis datanya. Hal ini terjadi karena saat ini, ABS masih belum dapat melakukan akses kedalam basis data sungguhan seperti yang sudah dipaparkan pada bagian batasan penelitian.\\

Selain batasan terkait akses basis data tersebut, terdapat pula keterbatasan lainya yang dimiliki oleh ABS MVC Framework dalam membuat komponen Controller yaitu adalah belum adanya mekanisme yang dapat digunakan oleh ABS MVC Framework untuk memberikan banyak model kedalam komponen View. Hal ini menyebabkan adanya keterbatasan pada komponen View seperti yang dibahas pada bagian evaluasi pembuatan komponen View. Adanya keterbatasan pada komponen Controller dalam memberikan banyak model adalah karena penulis belum menemukan pengganti tipe data \texttt{Generic} yang dimiliki oleh JAVA untuk ABS.\\

Walaupun terdapat beberapa keterbatasan fitur pada komponen Controller ABS MVC Framework, akan tetapi secara karakteristik antara komponen Controller ABS MVC Framework dengan Play Framework tidak jauh berbeda. Sehingga dapat disimpulkan bahwa komponen Controller yang dibentuk dengan menggunakan ABS MVC Framework masih sesuai dengan karakteristik yang dipaparkan oleh \cite{krasner1988desc} dan \cite{leff2001web} serta \textit{framework} MVC lain yang biasa digunakan.

\section{Evaluasi Proses Penanganan Perubahan Requirement}

Perubahan \textit{requirement} terhadap sistem, merupakan satu hal yang tidak dapat dihindari. Dalam konteks ABS MVC Framework, perubahan \textit{requirement} terssebut dapat ditangani dengan membuat kode delta seperti yang penulis bahas pada bab sebelumnya. Dengan menggunakan pendekatan ini, setiap perubahan \textit{requirement} akan tercatat dalam bentuk kode delta yang dibuat sehingga para pengembang memiliki dokumentasi yang jelas terhadap setiap perubahan yang terjadi di dalam sistem. Selain itu, dengan menggunakan pendekatan \textit{delta modelling}, para pengembang aplikasi web dapat dengan leluasa memilih perubahan mana saja yang ingin diimplementasikan tanpa harus kehilangan aplikasi yang aslinya.\\

Lain halnya dengan menggunakan Play Framework, pada saat terjadi perubahan \textit{requirement} pada sistem, mau tidak mau para pengembang aplikasi web tersebut harus melakukan perubahan secara langsung terhadap setiap komponen Model, View dan Controller-nya. Hal ini tentunya akan menyebabkan hilangnya kode aplikasi yang sudah dibuat sebelumnya. Dengan menggunakan pendekatan ini, para pengembang aplikasi web tersebut tidak dapat melacak jejak perubahan yang terjadi pada perangkat lunak yang dibuat dikarenakan setiap perubahan yang terjadi pada sistem langsung berdampak pada kode asli pada sistem tersebut.\\

Berdasarkan pemaparan diatas, dapat diketahui bahwa proses penangan perubahan \textit{requirement} dengan menggunakan pendekatan \textit{delta modelling} memberikan dampak yang lebih baik dibandingkan dengan melakukan perubahan secara langsung terhadap kode aplikasi yang dibuat. Dengan demikian, penulis berpendapat bahwa dalam penangan perubahan \textit{requirement} ini ABS MVC Framework memiliki nilai lebih dibandingkan dengan Play Framework.

\section{Evaluasi penerapan SPL dengan menggunakan ABS MVC Framework}

\textit{Software Product Line Engineering} (SPLE) merupakan sebuah paradigma yang digunakan dalam proses pengembangan perangkat lunak dengan menggunakan prinsip \textit{platform} dan \textit{mass customisation} \citep[p.~14]{pohl2005software}. Dengan menggunakan paradigma ini, para pengembang perangkat lunak dapat menciptakan banyak variasi produk dari produk inti (\textit{core product}) yang dibuat dengan cara melakukan proses perubahan (\textit{customization}) pada produk tersebut. Dalam konteks ABS MVC Framework, proses pembuatan SPLE dapat dilakukan dengan menggunakan pendekatan \textit{delta modelling}. Dengan menggunakan pendekatan ini, para pengembang perangkat lunak dapat menerapkan prinsip \textit{platform} dengan cara membuat \textit{core product} menggunakan kode ABS dan prinsip \textit{mass customization} dengan cara membuat banyak kode delta yang nantinya akan diterapkan di dalam \textit{core product} tersebut untuk dapat menghasilkan banyak variasi produk / \textit{productline}.\\

Sebagai contoh, pada pembahasan sebelumnya penulis telah membuat sebuah perubahan \textit{requirement} pada aplikasi katalog produk yang penulis buat. Perubahan \textit{requirement} tersebut dapat dianggap sebagai sebuah variasi produk dari produk intinya yaitu sebuah aplikasi katalog produk dengan tanpa kategori produk di dalamnya. Dalam kasus ini, penulis memiliki sebuah \textit{productline} yaitu aplikasi web katalog produk yang menyertakan kategori produk di dalamnya. Apabila penulis ingin membuat \textit{productline} yang lain seperti misalnya menambahkan atribut "diskon" pada data produk, penulis cukup membuat kode delta lain dan menambahkannya kedalam konfigurasi produk untuk dapat menghasilkan variasi produk yang baru.\\

Berdasarkan pemaparan tersebut, dapat diketahui bahwa ABS MVC Framework dapat memfasilitasi para pengembang perangkat lunak dalam menghasilkan SPL untuk aplikasi web yang dibuat.