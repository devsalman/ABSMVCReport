% Halaman Abstract
%
% @author  Andreas Febrian
% @version 1.00
%

\chapter*{ABSTRACT}

\vspace*{0.2cm}

\noindent \begin{tabular}{l l p{11.0cm}}
	Name&: & \penulis \\
	Program&: & \programEng \\
	Title&: & \judulInggris \\
\end{tabular} \\ 

\vspace*{0.5cm}

\noindent 
\\ Entering the 21\textsuperscript{st} century , the role of information technology , especially software continues to expand into each side of human life . With the wider penetration and public acceptance of the software , the software vendors competing to be able to produce software that can be used by the wider community. One of the challenges to be faced by the software vendors is about the extensive segmentation of the software market as a result of the wide variations of demand for such software. Therefore, it is necessary to have a solution that can be used by software vendors to create software product variations easily. In this study, the author uses an approach called Software Product Line Engineering (SPLE) with the help of Abstract Behavioral Specification (ABS) modeling language to make a web-based software product line using Model-View-Controller design pattern.

\vspace*{0.2cm}

\noindent Keywords: \\ 
\noindent Model-View-Controller (MVC), Software Product Line (SPL), Abstract Behavioural Specification, Web Application, delta modeling\\

\newpage