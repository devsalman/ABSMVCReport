\chapter{Landasan Teori}

%---------------------------------------------------------
\section{Model View Controller}
%---------------------------------------------------------
\noindent
\textit{Model-View-Controller} (MVC) atau yang biasa juga dikenal dengan sebutan \textit{Presentation-Abstraction-Control} (PAC) merupakan salah satu pendekatan dalam proses pengembangan perangkat lunak yang ditujukan untuk melakukan pemisahan antara logika aplikasi, data, dan presentasi. Konsep ini dibangun atas kesadaran bahwa sebuah model domain aplikasi yang sama dapat disajikan dan diperlakukan secara berbeda tergantung dari kebutuhan si pengguna aplikasi. Dengan menggunakan pendekatan ini, seorang pengembang perangkat lunak dapat berfokus pada satu bagian saja tanpa harus mengkhawatirkan akan terkena dampak perubahan ataupun memberikan perubahan ke bagian aplikasi lainnya.

\subsection{Sejarah Singkat MVC}
\noindent
Konsep MVC diterapkan pertama kalinya oleh Alan Kay, Dan Ingalls, dan Adele Goldberg pada tahun 1980 ketika mereka merancang bahasa pemrograman smalltalk-80 di Xerox PARC Learning Research Group (LRG) \citep{krasner1988desc}. bahasa pemrograman ini didesain dan dikembangkan dengan menggunakan strategi yang merepresentasikan informasi, tampilan, dan kontrol pada lingkungan pemrogramannya. Strategi ini digunakan dengan tujuan (1) untuk membuat kumpulan komponen sistem spesial yang dibutuhkan dalam mendukung proses pengembangan perangkat lunak yang interaktif serta (2) menyediakan kumpulan komponen sistem umum yang dapat membantu pengembang dalam menciptakan aplikasi grafis yang interaktif dengan mudah \citep{krasner1988desc}. Strategi dan tujuan tersebut dibuat dalam rangka menjawab isu utama dalam pengembangan perangkat lunak yaitu terkait pemanfaatan kembali komponen yang telah dibuat (\textit{reusability}) dan kemudahan dalam menggabungkan setiap komponen aplikasi (\textit{plugability}). \\

\noindent
Belajar dari pengalamannya dalam mengembangkan smalltalk-76, para pengembang smalltalk-80 menemukan bahwa untuk mencapai sebuah modularitas yang tinggi diperlukan adanya tiga buah pemisahan fokus dalam pengembangan aplikasi. Tiga buah pemisahan fokus tersebut antara lain adalah (1) memisahkan setiap komponen yang merepresentasikan model domain aplikasi dengan (2) cara yang digunakan untuk merepresentasikan model tersebut ke pengguna aplikasi dan (3) cara yang digunakan oleh pengguna dalam berinteraksi dengan model tersebut. Tiga buah pemisahan tersebut dapat terangkum dalam sebuah konsep yang disebut dengan \textit{Model-View-Controller} (MVC).

\subsection{Penerapan MVC dalam Pengembangan Aplikasi Web}
Aplikasi web merupakan aplikasi yang tergolong interaktif karena aplikasi jenis ini banyak memiliki elemen-elemen yang dapat digunakan untuk berinteraksi dengan penggunannya. Sebagai contoh, dalam sebuah halaman situs web tentunya kita akan menemukan banyak tombol, gambar, tautan, dan kotak isian yang dapat kita gunakan untuk berinteraksi dengan situs web tersebut. Untuk sebuah aplikasi yang tergolong interaktif, adanya pemisahan antara logika aplikasi, data, dan presentasi tentunya akan dapat meningkatkan fleksibilitas aplikasi tersebut dari segi pengembangan. \\

\noindent
Pada dasarnya, arsitektur apliksi berbasis web terbagi menjadi dua bagian yaitu \textit{client} dan \textit{server}. Dengan arsitektur yang seperti ini, para pengembang aplikasi tidak dapat menentukan dengan jelas bagaimana bentuk partisi yang harus dibuat untuk aplikasi tersebut. Sebagai contoh, dengan adanya pembagian antara \textit{client} dan \textit{server}, para pengembang aplikasi harus menentukan dimanakah komponen \textit{view} akan dibentuk? Apakah komponen ini akan dibentuk di tingkat \textit{client} ataukah di tingkat \textit{server}. Begitupun dengan komponen \textit{Model} dan \textit{Controller
}-nya. Apakah komponen-komponen tersebut akan akan dibuat di tingkat \textit{client}, \textit{server}, atau keduanya? Pada akhirnya, keputusan dalam menentukan skema partisi yang dipakai akan sangat bergantung pada teknologi yang digunakan \citep{leff2001web}. \\

\noindent
Permasalahan terkait pemisahan antara \textit{client} dan \textit{server} pada aplikasi berbasis web menjadikan penerapan MVC lebih sulit. Proses penerapan MVC akan dapat berhasil apabila (1) para pengembang aplikasi sudah mengetahui bagaimana skema partisi yang akan diterapkan serta (2) teknologi dan infrastruktur yang ada \textit{compatible} dengan skema partisi yang diterapkan. Oleh karena itu, perlu adanya sebuah pendekatan yang dapat digunakan oleh para pengembang untuk memastikan dua hal tersebut.

%---------------------------------------------------------
\section{Software Product Line Engineering (SPLE)}
%---------------------------------------------------------
\noindent
\textit{Software Product Line Engineering} (SPLE) merupakan sebuah paradigma yang digunakan dalam proses pengembangan perangkat lunak dengan menggunakan prinsip \textit{platform} dan \textit{mass customisation} \citep[p.~14]{pohl2005software}. Dalam industri perangkat lunak, istilah \textit{platform} atau \textit{software platform} biasa diartikan sebagai sebuah sistem komputer (misal: prosesor atau kombinasi antara perangkat keras dengan sistem operasi) yang menyebabkan dapat berjalannya sebuah program komputer. Sedangkan dalam konteks SPLE, yang dimaksud dengan \textit{platform} adalah sebuah subsistem dan \textit{interface} yang membentuk sebuah struktur umum dimana nantinya sebuah produk turunan dapat dikembangkan dan diproduksi secara efisien \citep[p.~15]{pohl2005software}. \\

\noindent
Dalam paradigma SPLE, proses pengembangan perangkat lunak dibagi menjadi dua bagian yaitu \textit{Domain Engineering} dan \textit{Application Engineering} \citep[p.~21]{pohl2005software}. \textit{Domain Engineering} adalah sebuah proses dalam SPLE dimana pada tahap ini seluruh \textit{commonality} dan \textit{variability} dari SPL didefinisikan dan direalisikan. Sedangkan tahap \textit{Application Engineering} adalah sebuah proses dimana aplikasi dari SPL dibuat dengan cara memanfaatkan \textit{domain artifact} yang telah dibuat pada tahap sebelumnya dan mengeksploitasi \textit{variability} yang ada di dalam SPL tersebut. Tahapan-tahapan proses dalam SPLE ini biasa disebut dengan istilah \textit{SPLE Framework}. \\

\begin{figure}
    \centering
    \includegraphics[width=0.8\textwidth]
        {img/sple-process.png}
    \caption{SPLE Framework}
\end{figure}
\vspace{-0.8cm}
\begin{center}
{\small Sumber gambar: \citep{pohl2005software}}
\end{center}

%---------------------------------------------------------
\section{Abstract Behavioural Spesification (ABS)}
%---------------------------------------------------------
\noindent
Abstract Behavioural Specification Language (ABS) merupakan sebuah bahasa pemodelan yang dibuat oleh konsorsium uni eropa di bawah proyek bernama \textit{Highly Adaptable and Trustworthy Software using Formal Method} (HATS). Tujuan dari proyek HATS dalam menciptakan ABS adalah untuk menciptakan sebuah pendekatan yang \textit{model-centric} dalam melakukan proses perancangan, implementasi dan verifikasi dari sebuah sistem yang \textit{highly-configurable} \citep{clarke2012variability}. Pada dasarnya ABS dibagi kedalam beberapa layer (lihat gambar 2.x) yang diantaranya adalah \textit{functional abstraction}, \textit{OO-Imperative layer}, \textit{Concurency Model} dan \textit{ABS Core}. \\

\begin{figure}
    \centering
    \includegraphics[width=0.6\textwidth]
        {img/abs-layers.png}
    \caption{ABS Layer}
\end{figure}
\vspace{-0.8cm}
\begin{center}
{\small Sumber gambar: \citep{hahnle2013hats}}
\end{center}

\noindent
Sebagai sebuah bahasa pemrograman \textit{imperative} yang menganut konsep \textit{Object Oriented}, secara umum ABS memiliki sintaks yang sama dengan bahasa pemrograman JAVA (walaupun lebih sederhana). Salah satu perbedaan yang paling mendasar antara ABS dengan JAVA adalah pada konsep \textit{code reuse}-nya. Pada bahasa pemrograman JAVA, konsep \textit{code reuse} diimplementasikan dengan cara membuat \textit{code inheritance} sedangkan pada ABS konsep tersebut diimplementasikan dalam betuk \textit{code deltas} \citep{hahnle2013hats}. \textit{code deltas} pada ABS merupakan sebuah kumpulan kode yang mendeskripsikan perubahan-perubahan kode pada kelas yang dituju. Dengan adanya konsep ini, ABS dapat melakukan manipulasi kelas seperti menambah atau menghilangkan \textit{variable} dan \textit{method}. \\

\noindent
Seperti yang sudah disebutkan sebelumnya bahwa di dalam ABS konsep \textit{code reuse} diimplementasikan dalam bentuk \textit{code deltas}. \textit{Code deltas} tersebut nantinya akan digunakan untuk memodelkan \textit{variability} yang terjadi di tingkat \textit{source code}. pemodelan \textit{variability} ini merupakan sebuah pendekatan yang dilakukan oleh ABS dalam membangun sebuah SPL. Proses pemodelan \textit{variability} ini biasa disebut juga sebagai proses \textit{Delta Modelling}. \\

\noindent
\textit{Delta Modelling} merupakan sebuah pendekatan yang fleksible dan modular dalam mewujudkan berbagai macam variasi produk dengan menggunakan kembali artifak-artifak yang ada \citep{hahnle2013hats}. Dalam proses \textit{Delta Modelling}, realisasi dari SPL dibentuk dari dua bagian yaitu \textit{core module} dan \textit{delta module}. \textit{Delta module} berisi fungsi-fungsi yang berlaku umum terhadap semua varian produk yang akan dibuat sedangkan \textit{delta modul} merupakan enkapsulasi dari perubahan-perubahan yang akan terjadi pada \textit{core product} untuk kemudian menghasilkan varian produk yang lain. \\

\begin{figure}
    \centering
    \includegraphics[width=0.6\textwidth]
        {img/delta-modelling-1.png}
\end{figure}

\begin{figure}
    \centering
    \includegraphics[width=0.6\textwidth]
        {img/delta-modelling-2.png}
    \caption{Delta Modelling pada ABS}
\end{figure}\vspace{-0.8cm}
\begin{center}
{\small Sumber gambar: \citep{clarke2012variability}}
\end{center}