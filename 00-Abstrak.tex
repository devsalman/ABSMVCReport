%
% Halaman Abstrak
%
% @author  Andreas Febrian
% @version 1.00
%

\chapter*{Abstrak}

\vspace*{0.2cm}

\noindent \begin{tabular}{l l p{10cm}}
	Nama&: & \penulis \\
	Program Studi&: & \program \\
	Judul&: & \judul \\
\end{tabular} \\ 

\vspace*{0.5cm}

\noindent 
\\ Seiring berjalannya waktu, komplesksitas dari \textit{requirement} sebuah aplikasi menjadi semakin meningkat dikarenakan semakian luasnya domain permasalahan yang harus di selesaikan oleh sebuah perangkat lunak. Hal ini tentunya menuntut sebuah perangkat lunak untuk terus berevolusi dalam memenuhi kebutuhan pengguna. Salah satu pendekatan yang digunakan untuk melakukan proses evolusi perangkat lunak adalah dengan menggunakan pendekatan \textit{Software Product Line Engineering} (SPLE). Untuk mewujudkan hal tersebut, terdapat sebuah teknologi yang bernama \textit{Abstract Behavioural Spesification} (ABS) yang dapat membantu pengembang dalam mengimplementasikan SPLE. Pada penelitian ini, penulis akan membuat sebuah \textit{framework} ABS yang dapat digunakan untuk membuat \textit{Software Product Line} (SPL) berbasis web dengan menerapkan pola \textit{Model-View-Controller} dan \textit{delta modelling} dalam realisasinya.

\vspace*{0.2cm}

\noindent Kata Kunci: \\ 
\noindent \textit{Model-View-Controller} (MVC), \textit{Software Product Line} (SPL), \textit{Abstract Behavioural Specification} (ABS), \textit{Web Application}, \textit{delta modelling}\\

\newpage