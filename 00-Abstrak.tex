%
% Halaman Abstrak
%
% @author  Andreas Febrian
% @version 1.00
%

\chapter*{Abstrak}

\vspace*{0.2cm}

\noindent \begin{tabular}{l l p{10cm}}
	Nama&: & \penulis \\
	Program Studi&: & \program \\
	Judul&: & \judul \\
\end{tabular} \\ 

\vspace*{0.5cm}

\noindent 
\\ Memasuki abad 21, peran teknologi informasi khususnya perangkat lunak terus meluas kedalam setiap sisi kehidupan manusia. Dengan semakin luasnya penetrasi dan penerimaan masyarakat terhadap perangkat lunak, para pengembang perangkat lunak berlomba-lomba untuk dapat menghasilkan perangkat lunak yang dapat digunakan oleh masyarakat luas. Salah satu tantangan yang harus dihadapi oleh para pengembang perangkat lunak adalah tentang banyaknya segmen pasar perangkat lunak akibat dari banyaknya variasi kebutuhan masyarakat terhadap perangkat lunak tersebut. Oleh karena itu dibutuhkan adanya solusi yang dapat digunakan oleh para pengembang perangkat lunak dalam menciptakan variasi-variasi produk perangkat luak dengan mudah. Dalam penelitian ini, Penulis mengguakan sebuah pendekatan bernama \textit{Software Product Line Engineering} dengan bantuan bahasa pemodelan Abstract Behavioral Specification (ABS) untuk dapat menghasilkan sebuah \textit{software product line} (SPL) berbasis web dengan menggunakna pola Model-View-Controller (MVC).

\vspace*{0.2cm}

\noindent Kata Kunci: \\ 
\noindent \textit{Model-View-Controller} (MVC), \textit{Software Product Line} (SPL), \textit{Abstract Behavioural Specification} (ABS), \textit{Web Application}, \textit{delta modeling}\\

\newpage