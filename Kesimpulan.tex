\chapter{Kesimpulan dan Penelitian Lebih Lanjut}

\section{Kesimpulan}
Berdasarkan hasil evaluasi yang penulis lakukan terhadap ABS MVC Framework, penulis membuat beberapa kesimpulan dari hasil penelitian ini yang diantaranya adalah:

\begin{enumerate}
    \item Pola pengembangan Model-View-Controller dapat diterapkan pada bahasa pemodelan ABS dengan menjadikan ABS sebagai komponen Model dan Controller-nya sedangkan HTML sebagai komponen View-nya. Dalam implementasinya, dibutuhkan adanya bantuan dari \textit{template engine} untuk mengintegrasikan komponen Model dengan komponen View yang dibuat.
    \item Fitur \textit{delta modeling} dapat diterapkan pada ABS MVC Framework dan digunakan untuk memodifikasi atau menambahkan komponen Model dan Controller pada \textit{framework} tersebut. Fitur \textit{delta modeling} tidak dapat diguanakan pada komponen View dikarenakan pada ABS MVC Framework komponen View tidak dibuat dengan menggunakan ABS melainkan HTML.
    \item Proses pengembangan SPL berbasis web dapat diterapkan dengan menggunakan ABS MVC Framework dengan cara menerapkan konsep \textit{delta modeling} pada aplikasi yang dibuat. Dalam konteks ABS MVC Framework, para pengembang perangkat lunak dapat membuat sebuah aplikasi web sebagai \textit{core product} dengan menggunakan ABS dan kemudian mebuat kode-kode delta serta konfigurasi produk yang nantinya akan diterapkan pada \textit{core product} yang dibuat untuk menghasilkan variasi produk yang diinginkan.
\end{enumerate}

\section{Penelitian Lebih Lanjut}
Berikut ini adalah beberapa topik penelitian yang dapat dilakukan untuk melnjutkan penelitian ini ketahap selanjutnya:

\begin{itemize}
    \item Melakukan integrasi ABS MVC Framework dengan Relational Database Management System (RDBMS) agar aplikasi yang dibuat dapat melakukan penyimpanan data secara \textit{persistent}. Proses integrasi ABS MVC Framework dengan RDBMS sangatlah penting mengingat saat ini kebutuhan akan adanya kemampuan penyimpanan aplikasi dalam sebuah aplikasi web sudah menjadi sebuah keharusan.
    \item Membuat \textit{web server} untuk ABS yang \textit{production ready}. Untuk saat ini \textit{web server} yang dibuat hanya untuk keperluan pengembangan saja sehingga banyak sekali kekurangan dan keterbatasan yang dimiliki oleh \textit{web server} tersebut sehingga tidak dapat digunakan untuk menjalankan aplikasi yang sudah memasuki tahap \textit{production}.
    \item Proses integrasi ABS MVC Framework dengan \textit{software testing framework} untuk meningkatkan kualitas perangkat lunak web yang dihasilkan oleh ABS MVC Framework.
\end{itemize}