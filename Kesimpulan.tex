\chapter{Kesimpulan}

Berdasarkan hasil evaluasi yang penulis lakukan terhadap ABS MVC Framework, penulis membuat beberapa kesimpulan yang diantaranya adalah:

\begin{enumerate}
    \item Pola pengembangan Model-View-Controller dapat diterapkan pada bahasa pemodelan ABS dengan menjadikan ABS sebagai komponen Model dan Controller-nya sedangkan HTML sebagai komponen View-nya. Dalam implementasinya, dibutuhkan adanya bantuan dari \textit{template engine} untuk mengintegrasikan komponen Model dengan komponen View yang dibuat.
    \item Fitur \textit{delta modelling} dapat diterapkan pada ABS MVC Framework dan digunakan untuk memodifikasi atau menambahkan komponen Model dan Controller pada \textit{framework} tersebut. Fitur \textit{delta modelling} tidak dapat diguanakan pada komponen View dikarenakan pada ABS MVC Framework komponen View tidak dibuat dengan menggunakan ABS melainkan HTML.
    \item Proses pengembangan SPL berbasis web dapat diterapkan dengan menggunakan ABS MVC Framework dengan cara menerapkan konsep \textit{delta modelling} pada aplikasi yang dibuat. Dalam konteks ABS MVC Framework, para pengembang perangkat lunak dapa membuat sebuah aplikasi web sebagai \textit{core product} dengan menggunakan ABS dan kemudian mebuat kode-kode delta dan konfigurasi produk yang nantinya akan diterapkan pada \textit{core product} yang dibuat untuk menghasilkan variasi produk yang diinginkan.
\end{enumerate}